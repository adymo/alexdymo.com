\documentclass[12pt]{letter}
\usepackage{resume}
\usepackage{hyperref}
\interlinepenalty 10000

\begin{document}

{\large\centering\textbf{Alexander Dymo}\\}

\address
{
  \texttt{alex@alexdymo.com} \\
  \url{http://www.alexdymo.com}\\
  \url{http://www.linkedin.com/in/adymo}\\
  +1\,312\,709\,3026 \\
  Chicago, USA \\
}


\begin{llist}

  %% Summary =========================================================
  \vspace{.30in}                %workaround

  \sectiontitle{Summary}

             Entrepreneur, Y Combinator alum, project manager, free software developer,
             programmer and engineer.

  %% Experience =========================================================
  \sectiontitle{Startup\\*Experience}

  \employer{Eligo Energy}
  \location{Chicago, IL}
  \position{Director of Engineering}
  \dates{Aug 2013 -- present}

  \startexperience

           \item Eligo Energy is retail energy supplier. We supply energy to both residential and commercial customers in deregulated markets.

           \item Eligo Energy supplies electricity to tens of thousands customers in Washington DC, Illinois, New York, Ohio and Maryland.

           \item I'm managing technology projects at Eligo. These include customer relationship management (CRM), billing, call center, online customer aquisition, partner portals and other software projects.

           \item I'm also coordinating Eligo's open source contributions. Our open source projects are ActiveConfig, Power Enum and X12. We also contribute to Fat Free CRM, Schema Plus, PG Power, NGINX Heroku Buildpack, Nanoc and other open source projects. We use open source to build our software systems and we do give back to the community. My goal is to always open source as much of our code as possible.

           \item As in any startup, my responsibilities do not end with software project management and development. I also work on online marketing campaigns, including CPC campaigns like Google AdWords and numerous incentive-driven campaigns.

  \endexperience

  \employer{Acunote}
  \location{Foster City, CA}
  \position{Director of Engineering}
  \dates{Jun 2008 -- Jun 2013}

  \startexperience

           \item Acunote is a project management and Scrum software made by Pluron, Inc.
           (DBA Acunote), a web startup based in Silicon Valley.

           \item Acunote's customers include IBM, EMC2, HP, Fujitsu, FCC, USGS, Havas,
           various startups such as Bump, PoundPay, Comprehend Systems and others.

           \item Now I'm directing the development of Acunote, leading the development team based in Ukraine.

           \item I was the first employee of Acunote (see below for details) and built the first releases of the product. When our startup got momentum and our Ukrainian team grew, I became the one to direct the team and oversee the development. After becoming a Director I got stock and attended Y Combinator as a co-founder.

           \item I hire and manage developers. I set up goals, plan tasks and monitor task completion. I track and improve team's performance, review team's work. I gather and document best practices, formalize development processes. And I still code.

           \item As a startup person, in addition to management and development I also do marketing. I worked on our Google AdWords and Yandex Direct campaigns, wrote and automated autoresponder emails, did promo website SEO.

  \endexperience

  \employer{Y Combinator}
  \location{Mountain View, CA}
  \position{Winter 2011 Batch}
  \dates{Jan 2011 -- Mar 2011}

  \startexperience

           \item In 2005, Y Combinator developed a new model of startup funding. Twice a year it invests a small amount of money in a large number of startups. The startups move to Silicon Valley for 3 months, during which Y Combinator works intensively with them to get the company into the best possible shape and refine their pitch to investors. Each cycle culminates in Demo Day, when the startups present to a large audience of investors. 

           \item My startup Acunote applied and got accepted into Y Combinator Winter 2011 batch. We spent 3 months in Mountain View (Silicon Valley) improving our product, company processes and metrics, marketing, PR, pitch to investors and more. We got investments from Y Combinator, Start Fund and others.

           \item We're proud to be a part of YC alumni network, probably the most valuable outcome of being a Y Combinator company.

  \endexperience

  \employer{Acunote}
  \location{Foster City, CA}
  \position{Principal Software Engineer}
  \dates{Jun 2006 -- Jun 2008}

  \startexperience

           \item I became the first employee of Acunote - startup based in Silicon Valley founded by Gleb Arshinov. Gleb had and idea for a online project management and Scrum software. I implemented that idea: designed the architecture, coded prototypes and first release myself.

           \item I learned how to develop web apps with Ruby on Rails and deploy them with Capistrano. I implemented asynchronous processing with Rinda. I used Puppet to automate server management and wrote my own scripts for other tasks. I set up Monit, Mongrel, Unicorn, PostgreSQL and other tools.

           \item I became a Ruby/Rails performance expert. We had to spend at least one month per year optimizing Acunote and we made Acunote fast, probably the fastest Rails app out there after Twitter. I presented what I know about Ruby/Rails performance at RailsConf, PgCon and various conferences in Ukraine.

           \item Acunote was the place where I learned what is it like to work in a startup. I solved fuzzy tasks, also non-technical ones. I talked to customers and did support. In short, jack of all trades.

           \item I worked from home. I organized my environment and time and made remote work efficient. Later the company hired more remote developers and I used my experience to organize their work.

  \endexperience

  \employer{Ki-Inform}
  \location{Mykolayiv, Ukraine}
  \position{CTO}
  \dates{Jan 2005 -- Apr 2006}

  \startexperience

           \item I co-founded the company together with two of my colleagues from the university. Ki-Inform does various types consulting: software consulting, business development, business process optimization and project management.

           \item  I was responsible for all types of software consulting work. I hired two employees and we developed small business automation software, installed and integrated application servers, performed software and hardware maintenance, set up company technical infrastructure.

           \item  At Ki-Inform I had to manage people for the first time. I learned how to plan projects, manage them to stay on schedule and meet time/cost estimates.

           \item  Eventually company shifted its focus to business consulting, but I wanted to work more on products. That's why I left Ki-Inform and started working on Acunote.

  \endexperience



  \sectiontitle{Free Software\\*Development\\*Experience}

  \employer{KDE, KDevelop}
  \location{}
  \position{Developer and Maintainer}
  \dates{2002 -- present}

  \startexperience

           \item KDevelop is a free open source IDE (Integrated Development Environment) for Linux, Max OS X and other Unix systems. KDevelop supports multiple programming languages and uses Kate as an editor. KDevelop can be extended with with plugins in C++, editor plugins in Javascript, and external shell scripts.

           \item It supports C/C++ out of the box and provides code highlighting, syntax and semantic checking,  as-you-type code completion, navigation to declarations, definitions and uses of namespaces, classes, functions and variables, project-wide code refactoring and more.

            \item Separate plugins add support for Ruby, Python, PHP and other programming languages.

            \item \url{http://www.kdevelop.org}\\*
                  \url{https://projects.kde.org/projects/extragear/kdevelop}

            \item I take pride in writing free software and I have fun. Everything I know about software development I learned by working on free software.

            \item I got my KDE developer account in 2002 for my work on Kugar - xml-based report generator that was a part of KOffice suite. I maintained Kugar for 2 years, improved its backend and wrote a report template designer called Kudesigner.

            \item I learned C++, Qt and KDE libs by working on Kugar, understood how to build graphical user interfaces and use   version control systems, learned best programming practices by looking over old-time KDE devs' shoulders.

            \item Since 2003 I've been developing KDevelop - KDE project's IDE. I started by fixing bugs and implementing features I needed. I liked what I did, got involved more and more and eventually became a project maintainer.

            \item As a project maintainer, I was responsible for product and strategy decisions, working with existing team members, mentoring new team members. I released all stable versions from 3.0.1 to 3.5.5 and several 4.0 betas.

            \item I am the co-author of current KDevelop core architecture and implementation (together with Andreas Pakulat). I designed and implemented KDevelop user interface for two recent 3.x and 4.x releases. I also wrote many plugins for KDevelop, including Ruby language support.

            \item KDevelop is my first large-scale project where I learned how to build robust architecture, design an API, work in a large team, interact with people, do release management and more.\\*
            \\*

  \endexperience

  \employer{KDE, KDevelop}
  \location{}
  \position{Google Summer of Code Mentor}
  \dates{2006 -- present}

  \startexperience

           \item In 2005 I participated in the Google Summer of Code program as a student. I wrote KDE-Eclipse, the set of Eclipse IDE plugins to make KDE application development easier.\\*
                \url{https://developers.google.com/open-source/soc/}\\*
                \url{http://websvn.kde.org/trunk/playground/devtools/eclipse/}

           \item Starting from 2006 I mentored other students working on KDevelop-related projects:\\*
                  1. 2006, C\# parser \/ integration for KDevelop by Jakob Petsovits\\*
                  2. 2006, A language agnostic refactoring API for kdevelop, by Tom Stephenson\\*
                  3. 2008, KDevelop DVCS(VCS) support by Evgeniy Ivanov\\*
                  4. 2012, Improve KDevelop Ruby Support by Miquel Sabate

            \item I also helped co-mentoring these projects:\\*
                  1. 2007, Python Support for KDevelop4 by Piyush Verma, mentored by Andreas Pakulat\\*
                  2. 2009, Static Code Visualization in KDevelop by Sandro Andrade, mentored by Aleix Pol

  \endexperience

  \employer{KDE, KDevelop}
  \location{}
  \position{Facebook Open Academy Mentor}
  \dates{2014 -- 2014}

  \startexperience

           \item Facebook Open Academy is a program designed to provide a practical, applied software engineering experience as part of a university student’s CS education. The program works closely with key faculty members at top CS universities to launch a course that matches students with active open source projects and mentors and allows them to receive academic credit for their contributions to the open source code base.

           \item I mentored and evaluated work of 3 students working on KDevelop IDE. My students worked on:\\*
                  - bug fixes in application wizard, outputview and grepview plugins\\*
                  - kdevelop-clang plugin features (C++ language support for KDevelop with CLang as backend): function and variable renaming refactoring, virtual override completion, implement function completion, and adjust signature helper

  \endexperience




  \sectiontitle{Academia\\*Experience}

  \employer{National University of Shipbuilding}
  \location{Mykolayiv, Ukraine}
  \position{Associate Professor}
  \dates{2002 -- present}

  \startexperience

            \item National University of Shipbuilding was the only one university in former Soviet Union that prepared students who built aircraft carriers. University's graduates worked at the Black Sea Shipyard in Mykolaiv, Ukraine - the only manufacturer of the Soviet/Russian aircraft carriers. Now the university diversified the curriculum. It teaches mechanical engineers, software engineers, managers that do not work in shipbuilding industry.

            \item I started part-time teaching at the university while working on my PhD.

            \item When I was a student, my people skills and presentation skills were bad. At the university I learned how to talk to people, how to get them do what I want, how to hold their attention. I learned to think clearly, put my thoughts onto paper and deliver them to the audience in a way people can understand.

            \item I taught many courses, but now I teach only two of them. I developed syllabus and created all course materials for those courses.

            \item Computational Linguistics (for Linguistics students)\\*
            This class helps non-technical linguistics students to understand what computers can and cannot do with the natural language. Students learn both linguistic (Chomsky) and statistic (Norvig) ways to do NLP.

            \item Artificial Intelligence and Expert Systems (for Linguistics students)\\*
            This class is an introduction into AI for non-technical students. They learn ideas behind AI, understand how modern AI systems work and which NLP techniques they use.

            \item Online course materials: \url{http://linguistics.mk.ua/}

  \endexperience


  %% Education ===========================================================
  \sectiontitle{Education}

  \school{National University of Shipbuilding}
  \location{Mykolayiv, Ukraine}
  \degree{PhD in Project Management (Operations Management)}
  \dates{2002 -- 2007}

  \startexperience

            \item PhD program was an ideal choice for me. I could do two things I liked in parallel: do research and continue writing open source software.

            \item PhD Dissertation: Enhancement of open source software project management efficiency

            \item Dissertation solves two problems of inefficient management of open source projects in the commercial companies. Even today most open source projects that originated in commercial setting are poorly managed. They also often fail. I identified two reasons for that.

            \item First, companies do not understand how open source teams work. To solve that, I created a model of open source software development lifecycle, studied self-organization in open source teams and developed a strategy to form and manage self-organizing teams.

            \item Second, companies do not understand how to estimate time and cost of open source project. To solve that, I developed an analog model for time and cost estimation.

            \item PhD thesis (in Russian):\\*
            \url{http://www.alexdymo.com/files/phd_thesis.pdf}
            \item PhD author's abstract (in Ukrainian):\\* \url{http://www.alexdymo.com/files/phd_author_abstract.pdf}

  \endexperience

  \school{National University of Shipbuilding}
  \location{Mykolayiv, Ukraine}
  \degree{PE in Computer Science}
  \dates{2000 -- 2002}

  \startexperience

            \item I decided that software engineering would become my profession and I wanted to get a formal degree too. So I applied for a 2 year computer science program in my university.

            \item I enjoyed most of the courses, but my favorite one was on automata theory and formal grammars.

            \item In addition to studying, I was programming most of my time. I had hobby projects and I had customers too. I developed two accounting applications for universities and sold it to 3 out of 5 universities in my city.

  \endexperience

  \school{National University of Shipbuilding}
  \location{Mykolayiv, Ukraine}
  \degree{MS in Energetics}
  \dates{2000 -- 2002}

  \startexperience

            \item I did masters because Ukrainians do not recognize Bachelor degree as the complete university degree.

            \item My research improves efficiency of marine waste heat recovery units. Modern marine engines are very efficient, so their flue gases are too cool for traditional heat recovery boilers to work optimally.

            \item My solution was to use thermosiphon heat exchanges instead of boilers. They have higher heat transfer capacity and work good even if the source of heat has low temperatures. The only problem is accurate calculation of their thermal performance.

            \item I developed a design procedure for marine thermosiphon-based waste heat recovery units and wrote a software to automate these calculations.

            \item Master's thesis (in Russian):\\*
            \url{http://www.alexdymo.com/files/master_thesis.pdf}

  \endexperience

  \school{National University of Shipbuilding}
  \location{Mykolayiv, Ukraine}
  \degree{BS in Energetics}
  \dates{1996 -- 2000}

  \startexperience

            \item In secondary school I was interested in history. I loved to dig through facts and understand why things happened the way they happened.

            \item However, I grew in a family of engineers and I simply knew I wanted to become an engineer too. In addition to history books, I studied math and bits of aerodynamics on my own. That's why I thought Mechanical engineering would be a perfect major for me.

            \item And for some time it was. My favorite courses were thermodynamics, fluid dynamics and heat exchange. At the same time I didn't enjoy anything related to shipbuilding.

            \item I was looking for another interesting things to do and found computer programming. I started by completing all assignments for my programming class in one night and ended up assisting my Computer Science professor in his commercial projects.

  \endexperience

\end{llist}
%%\layout
\end{document}

